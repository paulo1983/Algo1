
\documentclass{article}
	\usepackage[utf8]{inputenc}
	\usepackage[spanish]{babel}
	
	\begin{document}
		
		\section{Factorial}	

			\begin{document}
		
			
			
			El factorial de un entero positivo n, el factorial de n o n factorial se define en principio como el producto de todos los números
			enteros positivos desde 1 (es decir, los números naturales) hasta n. Por ejemplo, $\\$
			
			5! = 1 x 2 x 3 x 4 x 5 = 120 $\\$
			
			La operación de factorial aparece en muchas áreas de las matemáticas, particularmente en combinatoria y análisis matemático.
			De manera fundamental el factorial de n representa el número de formas distintas de ordenar n objetos distintos (elementos sin
			repetición). Este hecho ha sido conocido desde hace varios siglos, en el siglo XII por parte de estudiosos hindúes.
			
			
			La función factorial es formalmente definida mediante el producto (si n es mayor a 0)
			
	

				\subsection{Definicion}
				
				la fucnción factorial es formalmente definida mediante el producto (si n es mayor a 0) $\\$
				
				$\prod_{i = 5}{k} $ $\\$
				
				Una extensión común es:	$\\$
				
				0! = 1		
				
				
				$\sum{4}$
				
				
				\subsection{Especificacion}
				
				A continuación mostrar la especificación de la función factorial:
				
				TODO (agregaremos la especificación en un ejercicio posterior)
				
				\subsection{Implementacion recursiva}
				
				Aqui podemos ver la especificación recursiva
				
				\begin{verbatim}
					factorial n | n == 0 = 1
					            | n > 0  = n * factorial (n-1) 
				\end{verbatim}
				
		\end{document}
	
	\end{document}	
